%%%%%%%%%%%%%%%%%%%%%%%%%%%%%%%%%%%%%%%%%%%%%%%%%%%%%%%%%%
%                                                                                      %
%         Bristol Project LaTex Template            %
%                                                                                      %
%%%%%%%%%%%%%%%%%%%%%%%%%%%%%%%%%%%%%%%%%%%%%%%%%%%%%%%%%%
%
%   Author: Alex Charles           Email: aep.charles@gmail.com
%
% -----------------------------------------------------------------------------------
%      PACKAGES & OTHER DOCUMENT CONFIGURATIONS
% -----------------------------------------------------------------------------------
\documentclass[10pt]{article}
\usepackage[utf8]{inputenc}
\usepackage[T1]{fontenc}
\usepackage[british]{babel}
% ----------NEW BIBLATEX BIBLIOGRAPHY-----------------------------------------------
\usepackage[backend=bibtex,style = ieee]{biblatex} % Upgrades Bibliography

\addbibresource{BibFile.bib} %%% For biblatex
%e.g to add page number \footfullcite[chapter, p.~215]{AAIB}
% This allows can use footfullcite commands
% Note urldate field must be in yyyy-mm-dd to work - use online type
% Remeber to use \printbibliography in the footer
% -----------------------------------------------------------------------------------
\usepackage{sectsty}
\usepackage{amssymb,amsmath}
\usepackage{ifxetex,ifluatex}  %<<<<<<<<< Edit FONT HERE
\ifnum 0\ifxetex 1\fi\ifluatex 1\fi=0 % if pdftex
  \usepackage[T1]{fontenc}
  \usepackage[utf8]{inputenc}
\else % if luatex or xelatex
  \ifxetex
    \usepackage{mathspec}
    \setmainfont{Avenir-Light}
  \else
  % Font Package for XeLatex
    \usepackage{fontspec}
    \setmainfont{Avenir-Light}
  \fi
  \defaultfontfeatures{Ligatures=TeX,Scale=MatchLowercase}
\fi
\usepackage[fit]{truncate} %Truncates headers that are too long
\usepackage{fancyhdr}
\usepackage{lastpage}
\usepackage{extramarks}
\usepackage{gensymb}
\usepackage{lipsum}
\usepackage{float}
\usepackage{graphicx}
\graphicspath{{TempImg/}{Img/}}%<<<<<<<<< Location of Template Images and Other Images, Add folders here
\usepackage{subfig}
\usepackage{wrapfig}
\usepackage[font ={small,it}]{caption}
\usepackage{amsfonts,amsthm} % Math packages
% \usepackage{cite}
% \usepackage[maxlevel=3]{csquotes}
%    \MakeAutoQuote{‘}{’}
%    \MakeAutoQuote*{“}{”} %corrects quote marks
\usepackage{enumitem} % resume numbered lists
\usepackage{geometry}
\usepackage{booktabs} %<<<<<<<<< Table drawing package
\usepackage[table,xcdraw]{xcolor} %<<<<<<<<< Table drawing package
\usepackage{svg}
\usepackage{scrextend} %call footnotes
\usepackage[colorlinks, linkcolor = black, citecolor = black, filecolor = black, urlcolor = blue]{hyperref} % Creates Hyperlinks for references - add [colorlinks] for coloured hyperlinks
\usepackage{changepage} %Allows Adjust width to be used for the document (indenting paragraphs)
\usepackage{pdfpages} %Allows Pdfpages to be added to the document use \includepdf[pages={1}]{myfile.pdf}
\usepackage{pdflscape} %Change Pages from Portrait to Landscape

\usepackage[compact]{titlesec}
% \titlespacing*{\section}{0pt}{1.1\baselineskip}{\baselineskip}

\renewcommand*{\thefootnote}{\alph{footnote}} %%% Changes footnotes to letters
\usepackage[bottom]{footmisc} %%% Pushes footnote to bottom and to the margin

\DeclareCiteCommand{\footcite}[\mkbibfootnote]
{\usebibmacro{cite:init}%
\usebibmacro{prenote}}
{\usebibmacro{citeindex}%
\printtext[brackets]{\usebibmacro{cite:comp}}}
{\multicitedelim}
{\usebibmacro{cite:dump}%
\usebibmacro{postnote}}

\newenvironment{indentpara}{\begin{adjustwidth}{2cm}{}}{\end{adjustwidth}} %Declare adjust width wiht indentpara
\renewcommand{\labelitemii}{$\circ$}
\renewcommand{\labelitemiii}{$\diamond$}
\renewcommand{\labelitemiii}{$\cdot$}

% -----------------------------------------------------------------------------------
%                 Quotes
% -----------------------------------------------------------------------------------

\usepackage{epigraph}
% \epigraphsize{\small}% Default
\setlength\epigraphwidth{12cm}
\setlength\epigraphrule{0pt}

\usepackage{etoolbox}
\apptocmd{\sloppy}{\hbadness 10000\relax}{}{}%%%% > Removes Url bibliography warnings
\makeatletter
\patchcmd{\epigraph}{\@epitext{#1}}{\itshape\@epitext{#1}}{}{}
\makeatother

%%%% > For Quotes Use \epigraph{"Quote"}{ - \textup{Author}, Book}

% -----------------------------------------------------------------------------------
%                   NAMES & CLASS DEFINITION %<<<<<<<<< INSERT DETAILS HERE
% -----------------------------------------------------------------------------------
\newcommand{\AssignmentTitle}{Investigation of the Use Energy Storage Techniques to Reduce Peak Demand Charges for the University of Bristol}
\newcommand{\ModuleTitle}{Design Project 4 - Interim Report}
\newcommand{\University}{University of Bristol}
\newcommand{\Faculty}{Faculty of Engineering}
\newcommand{\UniCrest}{crestbris.png}
\newcommand{\UniLogo}{logobris.png}%<<<<<<<<< Make Sure Files are in the Template
%\newcommand{\GroupName}{Group 2}
\newcommand{\StudentNameA}{Alex Charles}
\newcommand{\StudentNumberA}{67634}
\newcommand{\SupervisorNameA}{Dr Theo Tryfonas}
\newcommand{\SupervisorEmailA}{Theo.Tryfonas@bristol.ac.uk}
% \newcommand{\SupervisorNameB}{Name}
% \newcommand{\SupervisorEmailB}{email@gmail.com}

% -----------------------------------------------------------------------------------
%        PACKAGES FOR MARKDOWN CONVERSION - FOR USE If Using Markdown to Latex
% -----------------------------------------------------------------------------------
\usepackage{fixltx2e} % provides \textsubscript
% use upquote if available, for straight quotes in verbatim environments
\IfFileExists{upquote.sty}{\usepackage{upquote}}{}
% use microtype if available
\IfFileExists{microtype.sty}{%
\usepackage{microtype}
\UseMicrotypeSet[protrusion]{basicmath} % disable protrusion for tt fonts
}{}
\hypersetup{unicode=true,
            pdftitle={\AssignmentTitle},
            pdfauthor={\StudentNameA},
            pdfborder={0 0 0},
            breaklinks=true}
\urlstyle{same}  % don't use monospace font for urls
\usepackage{fancyvrb}
\VerbatimFootnotes % allows verbatim text in footnotes
\usepackage{longtable,booktabs}
\IfFileExists{parskip.sty}{%
\usepackage{parskip}
}{% else
\setlength{\parindent}{0pt}s
\setlength{\parskip}{6pt plus 2pt minus 1pt}
}
\setlength{\emergencystretch}{3em}  % prevent overfull lines
\providecommand{\tightlist}{%
  \setlength{\itemsep}{0pt}\setlength{\parskip}{0pt}}
% \setcounter{secnumdepth}{0}
% Redefines (sub)paragraphs to behave more like sections
\ifx\paragraph\undefined\else
\let\oldparagraph\paragraph
\renewcommand{\paragraph}[1]{\oldparagraph{#1}\mbox{}}
\fi
\ifx\subparagraph\undefined\else
\let\oldsubparagraph\subparagraph
\renewcommand{\subparagraph}[1]{\oldsubparagraph{#1}\mbox{}}
\fi

% -----------------------------------------------------------------------------------
%                   WORD COUTNER - for XeLaTex
% -----------------------------------------------------------------------------------
\usepackage{xesearch}
\newcounter{words}
\newenvironment{counted}{%
  \setcounter{words}{0}
  \SearchList!{wordcount}{\stepcounter{words}}
    {a?,b?,c?,d?,e?,f?,g?,h?,i?,j?,k?,l?,m?,
    n?,o?,p?,q?,r?,s?,t?,u?,v?,w?,x?,y?,z?}
  \UndoBoundary{'}
  \SearchOrder{p;}}{%
  \StopSearching}

% -----------------------------------------------------------------------------------
%                   MARGINS, HEADERS & FOOTERS
% -----------------------------------------------------------------------------------
 \geometry{
 left=20mm,
 right=20mm,
 top=25mm,
 bottom=25mm,
 }
\linespread{1.15}

\pagestyle{fancy}
\lhead{\includegraphics[width = 0.2\textwidth]{\UniLogo}}
% \chead{\AssignmentTitle}
% \rhead{}
\lfoot{\StudentNameA}
\cfoot{}
\rfoot{Page \thepage} %%%% note the footer is swapped when page numbering style changes
\renewcommand\headrulewidth{0.4pt}
\renewcommand\footrulewidth{0.4pt}

\setlength\parindent{0pt}

\newcommand{\horrule}[1]{\rule{\linewidth}{#1}}

% -----------------------------------------------------------------------------------
%               DOCUMENT STRUCTURE COMMANDS
% -----------------------------------------------------------------------------------
% To sort out the formatting of header and footer when a page...
% ... split occurs "within" a problem environment.
\newcommand{\enterProblemHeader}[1]{
\nobreak\extramarks{#1 (Cont.)}\nobreak
\nobreak\extramarks{#1}{}\nobreak
}
% To sort out the formatting of header and footer when a page...
% ... split occur "between" problem environments.
\newcommand{\exitProblemHeader}[1]{
\nobreak\extramarks{#1 (Cont.)}\nobreak
\nobreak\extramarks{#1}{}\nobreak
}

% -----------------------------------------------------------------------------------
\begin{document}

  \setlength{\abovedisplayskip}{-18pt}
  \setlength{\belowdisplayskip}{0pt}
  \setlength{\abovedisplayshortskip}{-18pt}
  \setlength{\belowdisplayshortskip}{0pt}

%----------------------------------------------------------------------------------------
                                  %	TITLE PAGE FORMAT
%----------------------------------------------------------------------------------------
\pagenumbering{roman}
\begin{titlepage}

	\center % Center everything on the page
%----------------------------------------------------------------------------------------
%	HEADING SECTION
%----------------------------------------------------------------------------------------
		\usefont{OT1}{bch}{b}{n}
		\normalfont \normalsize \textsc{\University} \\ [10pt]
		\normalfont \normalsize \textsc{\Faculty} \\ [25pt]
%----------------------------------------------------------------------------------------
%	LOGO SECTION - Adds Univeristy Crest to the Report
%----------------------------------------------------------------------------------------
\newcolumntype{C}{>{\centering\arraybackslash} m{6cm} }  %# New column type
\begin{tabular}{CC}
  \includegraphics[width = 0.2\textwidth]{\UniCrest} &   \includegraphics[width=0.3\textwidth]{Arup_logo.png}
\end{tabular}\\[0.5cm]
%----------------------------------------------------------------------------------------
%	HEADING SECTION
%----------------------------------------------------------------------------------------
		\normalfont \normalsize \textsc{\ModuleTitle} \\ [25pt]
%----------------------------------------------------------------------------------------
%	TITLE SECTION
%----------------------------------------------------------------------------------------
		\horrule{0.5pt} \\[0.4cm]
		\huge \textbf{\AssignmentTitle} \\
		\horrule{2pt} \\[0.5cm]
%----------------------------------------------------------------------------------------
%	HEADING SECTION
%----------------------------------------------------------------------------------------
%		\normalfont \normalsize \textsc{\GroupName} \\ [25pt]
%----------------------------------------------------------------------------------------
%	AUTHOR SECTION
%----------------------------------------------------------------------------------------
\begin{minipage}{0.4\textwidth}
\begin{flushleft} \large
\emph{Supervisors:}\\
% Change Name
\textbf{\SupervisorNameA}\\
% \textbf{\SupervisorNameB}
\end{flushleft}
\end{minipage}
~
\begin{minipage}{0.4\textwidth}
\begin{flushright} \large
\emph{Email:} \\
\SupervisorEmailA\\
% \SupervisorEmailB

\end{flushright}
\end{minipage}\\[1cm]

\begin{minipage}{0.4\textwidth}
\begin{flushleft} \large
\emph{Author:}\\
	\textbf{\StudentNameA}
\end{flushleft}
\end{minipage}
~
\begin{minipage}{0.4\textwidth}
\begin{flushright} \large
\emph{Candidate Number:} \\
(\StudentNumberA)\\
\end{flushright}
\end{minipage}\\[2cm]

%----------------------------------------------------------------------------------------
%	DATE SECTION
%----------------------------------------------------------------------------------------
\textit{{\large \today}}\\[1cm] % Date, change the \today to a set date if you want to be precise
%----------------------------------------------------------------------------------------
\vfill % Fill the rest of the page with whitespace
\end{titlepage}

% \setcounter{page}{3}

\newpage


% -----------------------------------------------------------------------------------
%                             	 ABSTRACT
% -----------------------------------------------------------------------------------

\addcontentsline{toc}{section}{Abstract}
\begin{abstract}

\end{abstract}
% -----------------------------------------------------------------------------------
%                              TABLE OF CONTENTS
% -----------------------------------------------------------------------------------

\tableofcontents


% \newpage

\addcontentsline{toc}{section}{List of Tables}
\listoftables
\addcontentsline{toc}{section}{List of Figures}
\listoffigures
% \addcontentsline{toc}{section}{List of Acronyms}
% \section*{List of Acronyms}\label{acronyms}
% \textbf{BRB}: Be Right Back \\


% \newpage

% \addcontentsline{toc}{section}{Acknowledgements}
% \section*{Acknowledgements}\label{acknowledgements}

% \addcontentsline{toc}{section}{Declaration}
% \section*{Declaration}\label{declartion}
% I hereby declare that this report entitled “\AssignmentTitle” submitted to Bristol University, is a record of an original work completed by myself.\\

% \newpage

%% -----------------------------------------------------------------------------------
%%                          	  INTRODUCTION
%% -----------------------------------------------------------------------------------
\clearpage
\rfoot{Page \thepage\ of \pageref{LastPage}}
\pagenumbering{arabic}
\begin{counted} %<<<<<<<<<<<<<<STARTS WORD COUNTER
\section{Aims and Objectives}\label{aims-and-objectives}

Give a background to what the project is - done as a group

\newpage

\subsection{Individual Project Aim}\label{individual-project-aim}

The aim of this individual project is to investigate the feasibility of
a battery energy storage system to reduce peak energy demand charges for
a new University of Bristol campus. The University's peak demand charges
will be analysed and simulated, showing how peak demand is charged in a
normal use case. Different peak shaving system architectures will then
be added to the model, to find an optimum solution for the system's
design with respect to the system capital cost against peak demand
charge savings. A comparison between using a micro system, for room by
room use, or a macro system, being applied to the whole campus will be
conducted. The outputs of the project for 5\textsuperscript{th} year
will be a flexible model which produces an optimum peak shaving system
architecture for a given University scenario.

\subsection{Objectives}\label{objectives}

\textbf{Literature Review}

\begin{enumerate}
\item Perform a detailed literature review to investigate applicable peak shaving technologies based on similar use cases and performance. Research will include:
    \begin{itemize}
 \item a down-selection of different energy storage solutions, looking at their applicability to a University peak shaving system, comparing parameters such as; power-ratings , discharge times, charge times and costs.
\item investigating different prediction methods for peak demand surges, identifying limitations in current technology.
    \end{itemize}
\item Conduct research of different peak shaving systems, highlighting relevant modelling techniques and limitations.
\end{enumerate}

\textbf{Definition of System Architectures}

\begin{enumerate}[resume]
\item Define peak shaving system architectures, establishing the key performance variables.
\end{enumerate}

\textbf{Modelling and Analysis}

\begin{enumerate}[resume]
\item Analyse the University’s current peak demand charges. This will include understanding the University’s current demand charge structure and collecting typical energy usage data. Parameters such as time of day and sources of energy peaks will be included.
\item Produce a simulation to optimise the peak shaving system, comparing metrics including; unit cost and reduction in peak kWh charges based on University billing structure. This model will provide a comparative analysis between the different system architectures. This will detail savings against the University’s current peak demand charges. The model will be comprised of three stages:
\begin{enumerate}
\item A simulation of the University's a normal energy use case and peak demand charges, for use as a datum.
\item Inclusion of energy storage systems, simulating logic and detailing any prediction methods.
\item Addition of other factors to improve the model, taken from the literature review. This will take into other technologies such as peak load shedding to understand if peak demand can be further reduced.
\end{enumerate}
\item Analyse others areas where the system could deliver further benefits, such as AC to DC conversion, PV’s, micro-grids and increased sustainability.
\end{enumerate}

\textbf{Evaluation}

\begin{enumerate}[resume]
\item Evaluate results of the simulation, concluding on the effectiveness of different peak shaving system architectures against particular scenarios.
\end{enumerate}

\newpage

\section{Summary of Key Work and
References}\label{summary-of-key-work-and-references}

The following literature review outlines the key technologies required
for developing a optimised peak shaving system and conducting research
intended to justify and develop on the project objectives. Section
\ref{peak-demand-charges-and-loads} looks at pricing structures,
defining the problem which the system has to solve; highlighting the
limitations in current commercially available technology. Section
\ref{peak-shaving-technologies} explores the technologies that can be
used to define the system architecture. Finally, section
\ref{peak-shaving-systems---modelling-techniques} evaluates current
research of energy storage peak shaving systems, providing a reference
for shortlisting different system architectures.

\subsection{Peak Demand Charges and
Loads}\label{peak-demand-charges-and-loads}

\subsubsection{Pricing Structure}\label{pricing-structure}

\begin{itemize}
\tightlist
\item
  A charge based on a price per kW, typically the peak kW of the billing
  period\cite{schneiderRECPS}
\item
  Calculated using demand intervals, a short timeframe (often 15
  minutes) during which overall usage is aggregated and tracked as a
  total

  \begin{itemize}
  \tightlist
  \item
    The average calculated is the kW demand for this period
  \end{itemize}
\item
  Peak Shaving is the ability to control your usage from a supplier
  during intervals of high demand, in order to limit or reduce demand
  penalties for the billing period
\item
  After generation costs these are generally the second largest cost per
  kWh on an energy bill
\end{itemize}

\subsubsection{Current Peak Demand Management
Techniques}\label{current-peak-demand-management-techniques}

\begin{itemize}
\tightlist
\item
  To reduce peak demand, you will want to do electrical load planning
  and management. What this means, simply, is scheduling the use of
  electrical equipment to get the work done at the lowest possible
  electric load at any one time \cite{Reducing37:online}
\item
  Key two main methods:

  \begin{itemize}
  \tightlist
  \item
    Reducing usage through load shedding- understands which parts of the
    system can be switched off during periods of peak demand - these can
    be intelligent \cite{6199851}
  \item
    Adding capacity with on-site generation \cite{schneiderRECPS}
  \end{itemize}
\end{itemize}

\subsubsection{Typical Loads}\label{typical-loads}

\begin{itemize}
\tightlist
\item
  A major factor is how to handle surge loads. Surge loads occur when
  large electric motors are started. These large motors can draw ve
  times normal operating current for a short period of time. It is
  usually necessary to use some type of load starting device such as a
  soft starter or to start the motors sequentially \cite{baldorPS}.
\end{itemize}

\subsection{Peak Shaving Systems - Modelling
Techniques}\label{peak-shaving-systems---modelling-techniques}

As this project is assessing the feasibility of a electrical energy
storage system that reduces peak demand for the University, a review was
performed to understand research into peak shaving systems. Literature
around the subject is quite vast and varied, with products such as as
ABB's energy storage smart grid products, designed to perform load
levelling at grid level \cite{abbpeakshave}. However these technologies
are largely focused around supply levelling, using forecasting methods
and large battery storage systems to offset excess supply
\cite{5559470}, where micro level loads are ignored and paybacks periods
on EES systems are ignored. Supply levelling technologies on a macro
scale for homes can also be found in research, where the use large
batteries to reduce power fluctuations brought by the use of renewable
technologies. In \cite{Allik20161116}, shiftable water heating was used
as a the main storage device in houses, accounting for 50\% of household
electricity use. Excess load from wind turbines can be used to heat
water in these periods bypassing an inverter making a more efficient use
of the external energy. \cite{7564619}

\subsection{Peak Shaving Technologies}\label{peak-shaving-technologies}

\subsubsection{Electrical Energy Storage (EES)
System}\label{electrical-energy-storage-ees-system}

The chosen Electrical Energy Storage (EES) System is the main technology
which will effect the cost and feasibility of a peak shaving system.
where of these technologies are not suitable for a University peak
shaving system. An Electrical Energy Storage (EES) is the process of
converting electrical energy into a form that can be stored for later
use \cite{Chen2009291}. There are various energy storage technologies
available for large-scale applications which can be divided into four
kinds, i.e.~mechanical, electro-magnetic, chemical and electrochemical.

Due to the inherent properties of mechanical and These technologies into
different use cases, categorised as:

\begin{itemize}
\tightlist
\item
  \textbf{Energy Management:} for large scale storage, these are
  typically used by power plants for load leveling, ramping/load
  following, and spinning reserve.

  \begin{itemize}
  \tightlist
  \item
    PHS, CAES and CES are the typical technologies for high generation
    above 100MW - these methods are on a scale to large to be considered
    for this system
  \item
    Large-scale batteries, flow batteries, fuel cells, solar fuels, CES
    and TES are suitable for medium-scale energy management with a
    capacity of 10--100 MW, are appropriate for consideration for this
    system
  \end{itemize}
\item
  \textbf{Power quality:} For fast response times for power quality such
  as the instantaneous voltage drop, flicker mitiga- tion and short
  duration UPS

  \begin{itemize}
  \tightlist
  \item
    Flywheel, batteries, SMES, capacitor and super-capacitor millisecond
    response time lower than 1 MW - suitable perhaps in addition to
    large scale battery, flywheel efficieny to low for operational use.
  \end{itemize}
\item
  \textbf{Bridging power:} Relatively fast response (\textless{} 1 s)
  but also have relatively long discharge time (hours) The typical power
  rating for these types of applications is about 100 kW--10 MW.

  \begin{itemize}
  \tightlist
  \item
    Batteries, flow batteries, fuel cells and Metal-Air cells
    \cite{Chen2009291} By removing energy storage methods that would not
    be appropriate for the system, a table was created to compare the
    differing technologies. Due to only batteries and capacitors
    providing the response time and efficiencies required to make the
    system justifiable, only rechargeable batteries were compared. It is
    apparent from the literature review that the model of a University
    peak demand reduction system will need to compare different battery
    parameters along with their cost, in order to optimise the model.
  \end{itemize}
\end{itemize}

\subsubsection{Peak Demand Sensing}\label{peak-demand-sensing}

\begin{itemize}
\tightlist
\item
  Two systems, fixed scheduled and anticipated \cite{20164002874437} -
  good citation for university modelling
\item
  Online framework will continuously monitor the real-time demand
  reduction whilst actively perform peak demand reduction

  \begin{itemize}
  \tightlist
  \item
    Required supply from battery for the next interval is defined, in
    which to reduce the mean absolute percentage error (MAPE) it is
    computed by bringing the scheduled profile towards the forecasted
    supply, by an approximate factor
  \item
    Four zones in which the battery acts, that is charge, discharge,
    standby and idle. standby is where the battery is boost supplying
  \item
    Boundaries are set for predicted peak
  \item
    forecasted and present demand misalignment is reduced, fluctition
    along the threshold is minimised and energy is conserved only at an
    interval of potential peak demand.
  \item
    These use a method of taking power off the grid at an off peak time
  \item
    Thresholds are adjusted accordingly
  \item
    Both active and passive work , but peak reduction is much more
    consistent for active
  \item
    Where the real-time active control strategy achieved percentage
    reduction of up to 8.64\%, with accuracy of 70.52\% or percentage
    difference of 29.48\% away from the ideal reduction and has
    percentage performance of 77.08\% in achieving the peak demand
    reduction.
  \end{itemize}
\end{itemize}
% -----------------------------------------------------------------------------------
%                                  APENDIX
% -----------------------------------------------------------------------------------

\end{counted} %<<<<<<<<<<<<<<ENDS WORD COUNTER

\newpage
\section{Appendices}
Above were \thewords\ words. %<<<<<<<<<<<<<<DISPLAYS WORD COUNTER
% -----------------------------------------------------------------------------------
%                               BIBLIOGRAPHY - Insert Name of BIB File Here
% -----------------------------------------------------------------------------------
\newpage

% ---------------BIBTEX OLD-----------------------------------------------------
% \bibliographystyle{unsrt} %%%% Plain or alpha can change orders here
% \bibliography{BibFile}
% \nocite{*} %%%if you want to see all references even those note cited in the text
% -----------------------------------------------------------------------------------

\printbibliography

\end{document}
